%
% Copyright � 2017 Peeter Joot.  All Rights Reserved.
% Licenced as described in the file LICENSE under the root directory of this GIT repository.
%
%{
\input{../latex/blogpost.tex}
\renewcommand{\basename}{potentialMethods.tex}
%\renewcommand{\dirname}{notes/phy1520/}
\renewcommand{\dirname}{notes/ece1228-electromagnetic-theory/}
%\newcommand{\dateintitle}{}
%\newcommand{\keywords}{}

\input{../latex/peeter_prologue_print2.tex}

\usepackage{peeters_layout_exercise}
\usepackage{peeters_braket}
\usepackage{peeters_figures}
\usepackage{siunitx}
%\usepackage{mhchem} % \ce{}
\usepackage{macros_bm} % \BM
%\usepackage{macros_qed} % \qedmarker
%\usepackage{txfonts} % \ointclockwise

\beginArtNoToc

\generatetitle{A comparision of potential methods}
%\chapter{A comparision of potential methods}
%\label{chap:potentialMethods.tex}

\section{Motivation}

Here is a comparison of potential methods in the time domain for solutions of Maxwell's equations.  Following \citep{balanis2005antenna} Maxwell's equations are extended with magnetic sources and currents, as follows

\input{../ece1229-antenna/MaxwellsStatement.tex}

It will be assumed that the media is isotropic with \( \BB = \mu \BH \), and \( \BD = \epsilon \BE \).  The impedance of the media will be written as \( \eta = \sqrt{\mu/\epsilon} \), and the group velocity of a wave solution will be written as \( v = 1/\sqrt{\epsilon\mu} \).

\section{Traditional vector algebra}

Potential solutions to \cref{eqn:chapter3Notes:19} follow nicely by considering the electric and magnetic source cases separately, and then building a final solution with superposition.

\subsection{No magnetic sources}

When magnetic sources are omitted, it follows from \cref{eqn:chapter3Notes:80} that there is some \( \BA \) for which

%\begin{dmath}\label{eqn:potentialMethods:20}
\boxedEquation{eqn:potentialMethods:20}{
\BB = \spacegrad \cross \BA,
}
%\end{dmath}

Substitution into Faraday's law \cref{eqn:chapter3Notes:20} gives

\begin{dmath}\label{eqn:potentialMethods:40}
\spacegrad \cross \BE = - \PD{t}{}\lr{ \spacegrad \cross \BA },
\end{dmath}

or
\begin{dmath}\label{eqn:potentialMethods:60}
\spacegrad \cross \lr{ \BE + \PD{t}{ \BA } } = 0.
\end{dmath}

A gradient representation of this curled quantity, say \( -\spacegrad \phi \), will provide the required zero

%\begin{dmath}\label{eqn:potentialMethods:80}
\boxedEquation{eqn:potentialMethods:80}{
\BE = -\spacegrad \phi -\PD{t}{ \BA }.
}
%\end{dmath}

The final two Maxwell equations yield

\begin{dmath}\label{eqn:potentialMethods:100}
\begin{aligned}
-\spacegrad^2 \BA + \spacegrad \lr{ \spacegrad \cdot \BA } &= \mu \lr{ \BJ + \epsilon \PD{t}{} \lr{ -\spacegrad \phi -\PD{t}{ \BA } } } \\
\spacegrad \cdot \lr{ -\spacegrad \phi -\PD{t}{ \BA } } &= \rho/\epsilon,
\end{aligned}
\end{dmath}

or
%\begin{dmath}\label{eqn:potentialMethods:120}
\boxedEquation{eqn:potentialMethods:120}{
\begin{aligned}
\spacegrad^2 \BA - \inv{v^2} \PDSq{t}{ \BA }
- \spacegrad \lr{
\inv{v^2} \PD{t}{\phi}
+\spacegrad \cdot \BA
}
&= -\mu \BJ \\
\spacegrad^2 \phi + \PD{t}{} \lr{ \spacegrad \cdot \BA } &= -\rho/\epsilon.
\end{aligned}
}
%\end{dmath}

Note that the Lorentz condition \( \PDi{t}{(\phi/v^2)} + \spacegrad \cdot \BA = 0 \) can be imposed to decouple these, leaving non-homogeneous wave equations for the vector and scalar potentials respectively.

\subsection{No electric sources}

Without electric sources, a curl representation of the electric field can be assumed, satisfing Gauss's law

\boxedEquation{eqn:potentialMethods:140}{
\BD = - \spacegrad \cross \BF.
}

Substitution into the Maxwell-Faraday law gives
\begin{dmath}\label{eqn:potentialMethods:160}
\spacegrad \cross \lr{ \BH + \PD{t}{\BF} } = 0.
\end{dmath}

This is satisfied with any gradient, say, \( -\spacegrad \phi_m \), providing a potential representation for the magnetic field

%\begin{dmath}\label{eqn:potentialMethods:180}
\boxedEquation{eqn:potentialMethods:180}{
\BH = -\spacegrad \phi_m - \PD{t}{\BF}.
}
%\end{dmath}

The remaining Maxwell equations provide the required constraints on the potentials

\begin{subequations}
\label{eqn:potentialMethods:200}
\begin{dmath}\label{eqn:potentialMethods:220}
-\spacegrad^2 \BF + \spacegrad \lr{ \spacegrad \cdot \BF } = -\epsilon
\lr{
   -\BM - \mu \PD{t}{}
   \lr{
      -\spacegrad \phi_m - \PD{t}{\BF}
   }
}
\end{dmath}
\begin{dmath}\label{eqn:potentialMethods:240}
\spacegrad \cdot
\lr{
-\spacegrad \phi_m - \PD{t}{\BF}
}
= \inv{\mu} q_m,
\end{dmath}
\end{subequations}

or
%\begin{dmath}\label{eqn:potentialMethods:260}
\boxedEquation{eqn:potentialMethods:260}{
\begin{aligned}
\spacegrad^2 \BF - \inv{v^2} \PDSq{t}{\BF} - \spacegrad \lr{ \inv{v^2} \PD{t}{\phi_m} + \spacegrad \cdot \BF } &= -\epsilon \BM \\
\spacegrad^2 \phi_m + \PD{t}{}\lr{ \spacegrad \cdot \BF } &= -\inv{\mu} q_m.
\end{aligned}
}
%\end{dmath}

The general solution to Maxwell's equations is therefore
\begin{dmath}\label{eqn:potentialMethods:280}
\begin{aligned}
\BE &=
-\spacegrad \phi -\PD{t}{ \BA }
- \inv{\epsilon} \spacegrad \cross \BF \\
\BH &=
\inv{\mu} \spacegrad \cross \BA
-\spacegrad \phi_m - \PD{t}{\BF},
\end{aligned}
\end{dmath}

subject to the constraints \cref{eqn:potentialMethods:120} and \cref{eqn:potentialMethods:260}.

\section{GA with four-vectors}

%}
\EndArticle
