%
% Copyright � 2017 Peeter Joot.  All Rights Reserved.
% Licenced as described in the file LICENSE under the root directory of this GIT repository.
%
%{
\input{../latex/blogpost.tex}
\renewcommand{\basename}{potentialMethods.tex}
%\renewcommand{\dirname}{notes/phy1520/}
\renewcommand{\dirname}{notes/ece1228-electromagnetic-theory/}
%\newcommand{\dateintitle}{}
%\newcommand{\keywords}{}

\input{../latex/peeter_prologue_print2.tex}

\usepackage{peeters_layout_exercise}
\usepackage{peeters_braket}
\usepackage{peeters_figures}
\usepackage{siunitx}
%\usepackage{mhchem} % \ce{}
\usepackage{macros_bm} % \bcM
%\usepackage{macros_qed} % \qedmarker
%\usepackage{txfonts} % \ointclockwise

\newcommand{\lpartialt}[0]{{\stackrel{ \leftarrow }{\partial_t}}}
\newcommand{\rpartialt}[0]{{\stackrel{ \rightarrow }{\partial_t}}}

\beginArtNoToc

\generatetitle{A comparision of potential methods}
%\chapter{A comparision of potential methods}
%\label{chap:potentialMethods.tex}

\section{Motivation}

Here is a comparison of potential methods in the time domain for solutions of Maxwell's equations.  Following \citep{balanis2005antenna} Maxwell's equations are extended with magnetic sources and currents, as follows

\input{../ece1229-antenna/MaxwellsStatement.tex}

It will be assumed that the media is isotropic with \( \bcB = \mu \bcH \), and \( \bcD = \epsilon \bcE \).  The impedance of the media will be written as \( \eta = \sqrt{\mu/\epsilon} \), and the group velocity of a wave solution will be written as \( v = 1/\sqrt{\epsilon\mu} \).

\section{Traditional vector algebra}

Potential solutions to \cref{eqn:chapter3Notes:19} follow nicely by considering the electric and magnetic source cases separately, and then building a final solution with superposition.

\subsection{No magnetic sources}

When magnetic sources are omitted, it follows from \cref{eqn:chapter3Notes:80} that there is some \( \bcA \) for which

%\begin{dmath}\label{eqn:potentialMethods:20}
\boxedEquation{eqn:potentialMethods:20}{
\bcB = \spacegrad \cross \bcA,
}
%\end{dmath}

Substitution into Faraday's law \cref{eqn:chapter3Notes:20} gives

\begin{dmath}\label{eqn:potentialMethods:40}
\spacegrad \cross \bcE = - \PD{t}{}\lr{ \spacegrad \cross \bcA },
\end{dmath}

or
\begin{dmath}\label{eqn:potentialMethods:60}
\spacegrad \cross \lr{ \bcE + \PD{t}{ \bcA } } = 0.
\end{dmath}

A gradient representation of this curled quantity, say \( -\spacegrad \phi \), will provide the required zero

%\begin{dmath}\label{eqn:potentialMethods:80}
\boxedEquation{eqn:potentialMethods:80}{
\bcE = -\spacegrad \phi -\PD{t}{ \bcA }.
}
%\end{dmath}

The final two Maxwell equations yield

\begin{dmath}\label{eqn:potentialMethods:100}
\begin{aligned}
-\spacegrad^2 \bcA + \spacegrad \lr{ \spacegrad \cdot \bcA } &= \mu \lr{ \bcJ + \epsilon \PD{t}{} \lr{ -\spacegrad \phi -\PD{t}{ \bcA } } } \\
\spacegrad \cdot \lr{ -\spacegrad \phi -\PD{t}{ \bcA } } &= q_e/\epsilon,
\end{aligned}
\end{dmath}

or
%\begin{dmath}\label{eqn:potentialMethods:120}
\boxedEquation{eqn:potentialMethods:120}{
\begin{aligned}
\spacegrad^2 \bcA - \inv{v^2} \PDSq{t}{ \bcA }
- \spacegrad \lr{
\inv{v^2} \PD{t}{\phi}
+\spacegrad \cdot \bcA
}
&= -\mu \bcJ \\
\spacegrad^2 \phi + \PD{t}{} \lr{ \spacegrad \cdot \bcA } &= -q_e/\epsilon.
\end{aligned}
}
%\end{dmath}

Note that the Lorentz condition \( \PDi{t}{(\phi/v^2)} + \spacegrad \cdot \bcA = 0 \) can be imposed to decouple these, leaving non-homogeneous wave equations for the vector and scalar potentials respectively.

\subsection{No electric sources}

Without electric sources, a curl representation of the electric field can be assumed, satisfing Gauss's law

\boxedEquation{eqn:potentialMethods:140}{
\bcD = - \spacegrad \cross \bcK.
}

Substitution into the Maxwell-Faraday law gives
\begin{dmath}\label{eqn:potentialMethods:160}
\spacegrad \cross \lr{ \bcH + \PD{t}{\bcK} } = 0.
\end{dmath}

This is satisfied with any gradient, say, \( -\spacegrad \phi_m \), providing a potential representation for the magnetic field

%\begin{dmath}\label{eqn:potentialMethods:180}
\boxedEquation{eqn:potentialMethods:180}{
\bcH = -\spacegrad \phi_m - \PD{t}{\bcK}.
}
%\end{dmath}

The remaining Maxwell equations provide the required constraints on the potentials

\begin{subequations}
\label{eqn:potentialMethods:200}
\begin{dmath}\label{eqn:potentialMethods:220}
-\spacegrad^2 \bcK + \spacegrad \lr{ \spacegrad \cdot \bcK } = -\epsilon
\lr{
   -\bcM - \mu \PD{t}{}
   \lr{
      -\spacegrad \phi_m - \PD{t}{\bcK}
   }
}
\end{dmath}
\begin{dmath}\label{eqn:potentialMethods:240}
\spacegrad \cdot
\lr{
-\spacegrad \phi_m - \PD{t}{\bcK}
}
= \inv{\mu} q_m,
\end{dmath}
\end{subequations}

or
%\begin{dmath}\label{eqn:potentialMethods:260}
\boxedEquation{eqn:potentialMethods:260}{
\begin{aligned}
\spacegrad^2 \bcK - \inv{v^2} \PDSq{t}{\bcK} - \spacegrad \lr{ \inv{v^2} \PD{t}{\phi_m} + \spacegrad \cdot \bcK } &= -\epsilon \bcM \\
\spacegrad^2 \phi_m + \PD{t}{}\lr{ \spacegrad \cdot \bcK } &= -\inv{\mu} q_m.
\end{aligned}
}
%\end{dmath}

The general solution to Maxwell's equations is therefore
\begin{dmath}\label{eqn:potentialMethods:280}
\begin{aligned}
\bcE &=
-\spacegrad \phi -\PD{t}{ \bcA }
- \inv{\epsilon} \spacegrad \cross \bcK \\
\bcH &=
\inv{\mu} \spacegrad \cross \bcA
-\spacegrad \phi_m - \PD{t}{\bcK},
\end{aligned}
\end{dmath}

subject to the constraints \cref{eqn:potentialMethods:120} and \cref{eqn:potentialMethods:260}.

\subsection{Potential operator structure}

Knowing that there is a simple underlying structure to the potential representation of the electromagnetic field in the covarient formalism inspires the question of whether that structure can be found directly using the scalar and vector potentials determined above.

Specifically, forming

\begin{dmath}\label{eqn:potentialMethods:1020}
\bcF = \bcE + \eta I \bcH,
\end{dmath}

will result in the multivector potential equation

\begin{dmath}\label{eqn:potentialMethods:280b}
\bcF
=
-\spacegrad \phi -\PD{t}{ \bcA }
- \inv{\epsilon} \spacegrad \cross \bcK \\
+ I \eta
\lr{
\inv{\mu} \spacegrad \cross \bcA
-\spacegrad \phi_m - \PD{t}{\bcK}
}.
\end{dmath}

Can this be factored into into multivector operator and multivector potentials?  Expanding the cross products provides some direction

\begin{dmath}\label{eqn:potentialMethods:1040}
\begin{aligned}
\bcF
&=
- \PD{t}{ \bcA }
- \eta \PD{t}{I \bcK}
- \spacegrad \lr{ \phi - \eta I \phi_m } \\
&\quad + \frac{\eta}{2 \mu} \lr{ \rspacegrad \bcA - \bcA \lspacegrad }
+ \frac{1}{2 \epsilon} \lr{ \rspacegrad I \bcK - I \bcK \lspacegrad }.
\end{aligned}
\end{dmath}

Observe that the
gradient and the time partials can be grouped together

\begin{dmath}\label{eqn:potentialMethods:1060}
\begin{aligned}
\bcF
&=
- \PD{t}{ } \lr{\bcA + \eta I \bcK}
- \spacegrad \lr{ \phi + \eta I \phi_m }
+ \frac{v}{2} \lr{ \rspacegrad (\bcA + I \eta \bcK) - (\bcA + I \eta \bcK) \lspacegrad } \\
%=
%- \inv{v} \PD{t}{ (v \bcA) } + \frac{1}{2} \lr{ \rspacegrad (v \bcA) - (v \bcA) \lspacegrad }
%- \inv{v} \PD{t}{} (\eta v I \bcK)
%+ \frac{1}{2 } \lr{ \rspacegrad (\eta v I \bcK) - (\eta v I \bcK) \lspacegrad }
%- \spacegrad \lr{ \phi + \eta I \phi_m }
&=
\inv{2} \lr{
   \lr{ \rspacegrad - \inv{v} \rpartialt } \lr{ v \bcA + \eta v I \bcK }
   -
   \lr{ v \bcA + \eta v I \bcK} \lr{ \lspacegrad + \inv{v} \lpartialt }
} \\
&+\quad \inv{2} \lr{
   \lr{ \rspacegrad - \inv{v} \rpartialt } \lr{ -\phi - \eta I \phi_m }
   - \lr{ \phi + \eta I \phi_m } \lr{ \lspacegrad + \inv{v} \lpartialt }
}
% \\
%&=
%- \inv{2} \Biglr{
%   \lr{ \rspacegrad - \inv{v} \rpartialt } \lr{ - v \bcA - \eta v I \bcK +  \phi + \eta I \phi_m } \\
%   &\qquad -
%   \lr{ - v \bcA - \eta v I \bcK - \phi - \eta I \phi_m } \lr{ \lspacegrad + \inv{v} \lpartialt }
%}
,
\end{aligned}
\end{dmath}

or

%\begin{dmath}\label{eqn:potentialMethods:1080}
\boxedEquation{eqn:potentialMethods:1080}{
\begin{aligned}
\bcF
&=
\inv{2} \Biglr{
   \lr{ \rspacegrad - \inv{v} \rpartialt }
   \lr{
      - \phi
      + v \bcA
      + \eta I v \bcK
      - \eta I \phi_m
   } \\
   &\qquad -
   \lr{
      \phi
      + v \bcA
      + \eta I v \bcK
      + \eta I \phi_m
   }
   \lr{ \lspacegrad + \inv{v} \lpartialt }
}
.
\end{aligned}
}
%\end{dmath}

There's a conjugate structure to the potential on each side of the curl operation where we see a sign change for the scalar and pseudoscalar elements only.  The reason for this becomes more clear in the covariant formalism.

\section{Covarient formulation (GA)}

The GA formulation of Maxwell's equation is

\begin{dmath}\label{eqn:potentialMethods:300}
\lr{ \spacegrad + \inv{v} \PD{t}{} } \bcF
=
%\inv{\epsilon v}
\eta
\lr{ v q_\txte - \bcJ }
+ I \lr{ v q_\txtm - \bcM },
\end{dmath}

where \( \bcF \) is defined by \cref{eqn:potentialMethods:1020}.
This can be converted to covariant form, introducing a four-vector basis \( \setlr{ \gamma_\mu } \), where the spatial basis
\( \setlr{ \Be_k = \gamma_k \gamma_0 } \)
is reexpressed in terms of the Dirac basis \( \setlr{ \gamma_\mu } \).
By multiplying from the left with \( \gamma_0 \) a covariant form of Maxwell's equation is obtained

\begin{dmath}\label{eqn:potentialMethods:320}
\grad \bcF = \eta J - I M,
\end{dmath}

where
\begin{dmath}\label{eqn:potentialMethods:340}
\begin{aligned}
J &= \gamma^\mu J_\mu = ( v q_e, \bcJ ) \\
M &= \gamma^\mu M_\mu = ( v q_m, \bcM ) \\
\grad &= \gamma^\mu \partial_\mu = ( (1/v) \partial_t, \spacegrad ) \\
I &= \gamma_0 \gamma_1 \gamma_2 \gamma_3,
\end{aligned}
\end{dmath}

Here the metric choice is \( \gamma_0^2 = 1 = -\gamma_k^2 \).  Note that in this representation the electromagnetic field \( \bcF = \bcE + \eta I \bcH \) is a bivector, not a multivector as it is explicit (frame dependent) spacetime representation of \cref{eqn:potentialMethods:300}.

A potential representation can be obtained as before by considering electric and magnetic sources in sequence and using superposition to assemble a complete potential.

\subsection{No magnetic sources}

Without magnetic sources, Maxwell's equation splits into vector and trivector terms of the form

\begin{subequations}
\label{eqn:potentialMethods:360}
\begin{dmath}\label{eqn:potentialMethods:380}
\grad \cdot \bcF = \eta J
\end{dmath}
\begin{dmath}\label{eqn:potentialMethods:400}
\grad \wedge \bcF = 0.
\end{dmath}
\end{subequations}

A (spacetime) curl representation of the field will satisfy \cref{eqn:potentialMethods:400} allowing an immediate potential solution

\boxedEquation{eqn:potentialMethods:560}{
\begin{aligned}
&\bcF = \grad \wedge P \\
&\grad^2 P - \grad \lr{ \grad \cdot P } = \eta J.
\end{aligned}
}

This can be put into correspondence with \cref{eqn:potentialMethods:120} by noting that

\begin{dmath}\label{eqn:potentialMethods:460}
\begin{aligned}
\grad^2 &= (\gamma^\mu \partial_\mu) \cdot (\gamma^\nu \partial_\nu)  = \inv{v^2} \partial_{tt} - \spacegrad^2 \\
\gamma_0 P &= \gamma_0 \gamma^\mu P_\mu = P_0 + \Be_k P_k = P_0 + \bcP \\
\gamma_0 \grad &= \gamma_0 \gamma^\mu \partial_\mu = \inv{v} \partial_t + \spacegrad \\
\grad \cdot P &= \partial_\mu P^\mu = \inv{v} \partial_t P_0 - \spacegrad \cdot \bcP,
\end{aligned}
\end{dmath}

so multiplying from the left with \( \gamma_0 \) gives

\begin{dmath}\label{eqn:potentialMethods:480}
\lr{ \inv{v^2} \partial_{tt} - \spacegrad^2 } \lr{ P_0 + \bcP } - \lr{ \inv{v} \partial_t + \spacegrad }\lr{ \inv{v} \partial_t P_0 - \spacegrad \cdot \bcP } = \eta( v q_e - \bcJ ),
\end{dmath}

or

\begin{subequations}
\label{eqn:potentialMethods:500}
\begin{dmath}\label{eqn:potentialMethods:520}
\lr{ \inv{v^2} \partial_{tt} - \spacegrad^2 } \bcP - \spacegrad \lr{ \inv{v} \partial_t P_0 - \spacegrad \cdot \bcP } = -\eta \bcJ
\end{dmath}
\begin{dmath}\label{eqn:potentialMethods:540}
\spacegrad^2 P_0 - \inv{v} \partial_t \lr{ \spacegrad \cdot \bcP } = -q_e/\epsilon.
\end{dmath}
\end{subequations}

So \( P_0 = \phi \) and \( -\ifrac{\bcP}{v} = \bcA \), or

\boxedEquation{eqn:potentialMethods:600}{
P = \gamma_0\lr{ \phi - v \bcA }.
}

\subsection{No electric sources}

Without electric sources, Maxwell's equation now splits into

\begin{subequations}
\label{eqn:potentialMethods:620}
\begin{dmath}\label{eqn:potentialMethods:640}
\grad \cdot \bcF = 0
\end{dmath}
\begin{dmath}\label{eqn:potentialMethods:660}
\grad \wedge \bcF = -I M.
\end{dmath}
\end{subequations}

Here the dual of a (spacetime) curl yields a solution

%\begin{dmath}\label{eqn:potentialMethods:680}
\boxedEquation{eqn:potentialMethods:680}{
\bcF = I ( \grad \wedge Q ).
}
%\end{dmath}

Substituting this gives

\begin{subequations}
\label{eqn:potentialMethods:700}
\begin{dmath}\label{eqn:potentialMethods:720}
0
=
\grad \cdot (I ( \grad \wedge Q ) )
=
\gpgradeone{ \grad I ( \grad \wedge Q ) }
=
-I \grad \wedge ( \grad \wedge Q ).
\end{dmath}
\begin{dmath}\label{eqn:potentialMethods:740}
-I M
=
\grad \wedge (I ( \grad \wedge Q ) )
=
\gpgradethree{ \grad I ( \grad \wedge Q ) }
=
-I \grad \cdot ( \grad \wedge Q ).
\end{dmath}
\end{subequations}

The \( \grad \cdot \bcF \) relation \cref{eqn:potentialMethods:720} is identically zero as desired, leaving

%\begin{dmath}\label{eqn:potentialMethods:760}
\boxedEquation{eqn:potentialMethods:760}{
\grad^2 Q - \grad \lr{ \grad \cdot Q }
=
M.
}
%\end{dmath}

So the general solution with both electric and magnetic sources is

%\begin{dmath}\label{eqn:potentialMethods:800}
\boxedEquation{eqn:potentialMethods:800}{
\bcF = \grad \wedge P + I (\grad \wedge Q),
}
%\end{dmath}

subject to the constraints of \cref{eqn:potentialMethods:560} and \cref{eqn:potentialMethods:760}.  As before the four-potential \( Q \) can be put into correspondance with the conventional scalar and vector potentials by left multiplying with \( \gamma_0 \), which gives

\begin{dmath}\label{eqn:potentialMethods:820}
\lr{ \inv{v^2} \partial_{tt} - \spacegrad^2 } \lr{ Q_0 + \bcQ } - \lr{ \inv{v} \partial_t + \spacegrad }\lr{ \inv{v} \partial_t Q_0 - \spacegrad \cdot \bcQ } = v q_m - \bcM,
\end{dmath}

or
\begin{subequations}
\label{eqn:potentialMethods:840}
\begin{dmath}\label{eqn:potentialMethods:860}
\lr{ \inv{v^2} \partial_{tt} - \spacegrad^2 } \bcQ - \spacegrad \lr{ \inv{v} \partial_t Q_0 - \spacegrad \cdot \bcQ } = - \bcM
\end{dmath}
\begin{dmath}\label{eqn:potentialMethods:880}
\spacegrad^2 Q_0 - \inv{v} \partial_t \spacegrad \cdot \bcQ = -v q_m.
\end{dmath}
\end{subequations}

Comparing with \cref{eqn:potentialMethods:260} shows that \( Q_0/v = \mu \phi_m \) and \( -\ifrac{\bcQ}{v^2} = \mu \bcK \), or

\boxedEquation{eqn:potentialMethods:900}{
Q = \gamma_0 \eta \lr{ \phi_m - v \bcK }.
}

\subsection{Potential operator structure}

Observe that there is an underlying uniform structure of the differential operator that acts on the potential to produce the electromagnetic field.  Expressed as a linear operator of the
gradient and the potentials, that is

\( \bcF = L(\lrgrad, P, Q) \)

\begin{dmath}\label{eqn:potentialMethods:980}
\bcF
=
L(\grad, P, Q)
= \grad \wedge P + I (\grad \wedge Q)
=
\inv{2} \lr{ \rgrad P - P \lgrad }
+ \frac{I}{2} \lr{ \rgrad Q - Q \lgrad }
=
\inv{2} \lr{ \rgrad P - P \lgrad }
+ \frac{1}{2} \lr{ -\rgrad I Q - I Q \lgrad }
=
\inv{2} \lr{ \rgrad (P -I Q) - (P + I Q) \lgrad }
,
\end{dmath}

or
%\begin{dmath}\label{eqn:potentialMethods:1000}
\boxedEquation{eqn:potentialMethods:1000}{
\bcF
=
\inv{2} \lr{ \rgrad (P -IQ) - (P - IQ)^\dagger \lgrad }
%=
%\inv{2} \lr{ \grad (P -IQ) - (\grad (P - I Q))^\dagger }
.
}
%\end{dmath}

Alternatively, the potential can be written as a one sided operation using grade selection

\begin{dmath}\label{eqn:potentialMethods:1240}
\bcF = \gpgradetwo{ \grad ( P - I Q ) }.
\end{dmath}

Observe that \cref{eqn:potentialMethods:1000} can be
put into correspondance with \cref{eqn:potentialMethods:1080} using a factoring of unity \( 1 = \gamma_0 \gamma_0 \)

\begin{dmath}\label{eqn:potentialMethods:1100}
\bcF
=
\inv{2} \lr{ (-\rgrad \gamma_0) (-\gamma_0 (P -IQ)) - ((P + IQ) \gamma_0)(\gamma_0 \lgrad) },
\end{dmath}

where

\begin{subequations}
\label{eqn:potentialMethods:1120}
\begin{dmath}\label{eqn:potentialMethods:1140}
-\grad \gamma_0
=
-(\gamma^0 \partial_0 + \gamma^k \partial_k) \gamma_0
=
-\partial_0 - \gamma^k \gamma_0 \partial_k
=
\spacegrad
-\inv{v} \partial_t
,
\end{dmath}
\begin{dmath}\label{eqn:potentialMethods:1160}
\gamma_0 \grad
=
\gamma_0 (\gamma^0 \partial_0 + \gamma^k \partial_k)
=
\partial_0 - \gamma^k \gamma_0 \partial_k
=
\spacegrad
+ \inv{v} \partial_t
,
\end{dmath}
\end{subequations}

and
\begin{subequations}
\label{eqn:potentialMethods:1180}
\begin{dmath}\label{eqn:potentialMethods:1200}
-\gamma_0 ( P - I Q )
=
-\gamma_0 \gamma_0 \lr{ \phi -v \bcA + \eta I \lr{ \phi_m - v \bcK } }
=
-\lr{ \phi -v \bcA + \eta I \phi_m - \eta v I \bcK }
=
- \phi
+ v \bcA
+ \eta v I \bcK
- \eta I \phi_m
\end{dmath}
\begin{dmath}\label{eqn:potentialMethods:1220}
( P + I Q )\gamma_0
=
\lr{ \gamma_0 \lr{ \phi -v \bcA } + I \gamma_0 \eta \lr{ \phi_m - v \bcK } } \gamma_0
=
\phi + v \bcA + I \eta \phi_m + I \eta v \bcK
=
\phi
+ v \bcA
+ \eta v I \bcK
+ \eta I \phi_m
,
\end{dmath}
\end{subequations}

This recovers \cref{eqn:potentialMethods:1080} as desired.

\section{Non-covarient formulation (GA)}

When Maxwell equation is in the non-covariant form \cref{eqn:potentialMethods:300}, it is not obvious how to directly introduce a potential representation without first resorting to unpacking that equation into its vector convential form, or obtaining that potential directly from the covariant representation.  Determining the potential representation of the field in both the conventional and covariant formalisms started from observations that a wedge product was zero when one of the electric or magnetic sources was zero.  There must be some generalization of the curl operation that applies to the multivector spacetime derivative operator of \cref{eqn:potentialMethods:300}.

Given that four vectors have a multivector representation in \cref{eqn:potentialMethods:300}, and the source terms sum to a multivector with all grades, it is not unreasonable to expect that the potential will be a multivector with all grades.  This was seen explicitly when the covarient potential relation of \cref{eqn:potentialMethods:1000} was computed, where that four-potential multivector was found to be

\begin{dmath}\label{eqn:potentialMethods:920}
A = \phi - v \bcA - I \eta \lr{ \phi_m - v \bcK }.
\end{dmath}

The fact that a multivector potential (including all grades) is desired could also be guessed from the results of the conventional vector algebra potential found in \cref{eqn:potentialMethods:280}.  Such a guess is not entirely satifactory, since it does not easily yeild the structure of the differential operator on this potential.  In general, the expectation is that the potential will be of the form
%however, such a guess is not entirely satisfactory.  It is perhaps only a natural guess if the covariant GA formalism
%is known, and it does not provide the structure of the differential operator on this potential.  Suppose that it is possible to derive the electromagnetic field from a potential given by

\begin{dmath}\label{eqn:potentialMethods:960}
A = \sum_{k=0}^3 A_k,
\end{dmath}

where \( A_k \) is of grade \( k \).  However, obtaining the electromagnetic field from this potential could have a structure potentially as general as

\begin{dmath}\label{eqn:potentialMethods:940}
\bcF =
\inv{v} \PD{t}{}
\sum_{k=0}^3 a_k
A_k
+ \spacegrad (b A_0 + c A_3) + \sum_{k=1}^2 \lr{ d_k \rspacegrad A_k + e_k A_k \lspacegrad }.
\end{dmath}

Note that there are less degress of freedom possible for the gradients of the scalar and pseudoscalar potential components since those both commute with the gradient.
The constants \( a_k, b, c, d_k, e_k \) could be determined by comparing this presumed representation with
\cref{eqn:potentialMethods:1040}.

TODO: figure out how \cref{eqn:potentialMethods:1080} can be obtained directly from Maxwell's equation without resorting to unpacking that into its components or resorting to the equivalent covariant formalism.  Contrast this to the (gauge specific) single sided potential representation that Mauro found.

%}
\EndArticle
