%
% Copyright © 2017 Peeter Joot.  All Rights Reserved.
% Licenced as described in the file LICENSE under the root directory of this GIT repository.
%

Geometric algebra (GA) allows for a compact description of Maxwell's equations in either an explicit 3D representation or a STA (SpaceTime Algebra
\citep{doran2003gap}
) representation.  The 3D GA and
STA
representations Maxwell's equation both the form

\begin{dmath}\label{eqn:potentialMethods:1280}
L \bcF = J,
\end{dmath}

where \( J \) represents the sources,
\( L \) is a multivector gradient operator that includes partial derivative operator components for each of the space and time coordinates, and

\begin{dmath}\label{eqn:potentialMethods:1020}
\bcF = \bcE + \eta I \bcH,
\end{dmath}

is an electromagnetic field multivector, \( I = \Be_1 \Be_2 \Be_3 \) is the \R{3} pseudoscalar, and \( \eta = \sqrt{\mu/\epsilon} \) is the impedance of the media.

When Maxwell's equations are extended to include magnetic sources in addition to conventional electric sources (as used in antenna-theory \citep{balanis2005antenna} and microwave engineering \citep{pozar2009microwave}), they take the form

\input{../ece1229-antenna/MaxwellsStatement.tex}

The corresponding GA Maxwell equations in their respective 3D and STA forms are

\begin{subequations}
\label{eqn:potentialMethods:1300}
\begin{dmath}\label{eqn:potentialMethods:300}
\lr{ \spacegrad + \inv{v} \PD{t}{} } \bcF
=
%\inv{\epsilon v}
\eta
\lr{ v q_\txte - \bcJ }
+ I \lr{ v q_\txtm - \bcM }
\end{dmath}
\begin{dmath}\label{eqn:potentialMethods:320}
\grad \bcF = \eta J - I M,
\end{dmath}
\end{subequations}

where the wave group velocity in the medium is \( v = 1/\sqrt{\epsilon\mu} \), and the medium is isotropic with
\( \bcB = \mu \bcH \), and \( \bcD = \epsilon \bcE \).  In the STA representation, \( \grad, J, M \) are all four-vectors, the specific meanings of which will be spelled out below.

How to determine the potential equations and the field representation using the conventional distinct Maxwell's \cref{eqn:chapter3Notes:19} is well known.  The basic procedure is to consider the electric and magnetic sources in turn, and observe that in each case one of the electric or magnetic fields must have a curl representation.  The STA approach is similar, except that it can be observed that the field must have a four-curl representation for each type of source.  In the explicit 3D GA formalism
\cref{eqn:potentialMethods:300} how to formulate a natural potential representation is not as obvious.  There is no longer an reason to set any component of the field equal to a curl, and the representation of the four curl from the STA approach is awkward.  Additionally, it is not obvious what form gauge invariance takes in the 3D GA representation.

%\paragraph{New ideas (or believed to be) in this paper}
\paragraph{Ideas explored in these notes}

\begin{itemize}
\item GA representation of Maxwell's equations including magnetic sources.
\item STA GA formalism for Maxwell's equations including magnetic sources.
\item Explicit form of the GA potential representation including both electric and magnetic sources.
\item Demonstration of exactly how the 3D and STA potentials are related.
\item Explore the structure of gauge transformations when magnetic sources are included.
\item Explore the structure of gauge transformations in the 3D GA formalism.
\item Specify the form of the Lorentz gauge in the 3D GA formalism.
\end{itemize}

