%
% Copyright © 2017 Peeter Joot.  All Rights Reserved.
% Licenced as described in the file LICENSE under the root directory of this GIT repository.
%
Maxwell's equation in its explicit 3D form \cref{eqn:potentialMethods:300} can be
converted to STA form, by introducing a four-vector basis \( \setlr{ \gamma_\mu } \), where the spatial basis
\( \setlr{ \Be_k = \gamma_k \gamma_0 } \)
is expressed in terms of the Dirac basis \( \setlr{ \gamma_\mu } \).
By multiplying from the left with \( \gamma_0 \) a STA form of Maxwell's equation
\cref{eqn:potentialMethods:320}
is obtained,
where
\begin{equation}\label{eqn:potentialMethods:340}
\begin{aligned}
J &= \gamma^\mu J_\mu = ( v q_e, \bcJ ) \\
M &= \gamma^\mu M_\mu = ( v q_m, \bcM ) \\
\grad &= \gamma^\mu \partial_\mu = ( (1/v) \partial_t, \spacegrad ) \\
I &= \gamma_0 \gamma_1 \gamma_2 \gamma_3,
\end{aligned}
\end{equation}

Here the metric choice is \( \gamma_0^2 = 1 = -\gamma_k^2 \).  Note that in this representation the electromagnetic field \( \bcF = \bcE + \eta I \bcH \) is a bivector, not a multivector as it is explicit (frame dependent) 3D representation of \cref{eqn:potentialMethods:300}.

A potential representation can be obtained as before by considering electric and magnetic sources in sequence and using superposition to assemble a complete potential.
\subsection{No magnetic sources}
Without magnetic sources, Maxwell's equation splits into vector and trivector terms of the form
\begin{subequations}
\label{eqn:potentialMethods:360}
\begin{equation}\label{eqn:potentialMethods:380}
\grad \cdot \bcF = \eta J
\end{equation}
\begin{equation}\label{eqn:potentialMethods:400}
\grad \wedge \bcF = 0.
\end{equation}
\end{subequations}

A four-vector curl representation of the field will satisfy \cref{eqn:potentialMethods:400} allowing an immediate potential solution
\boxedEquation{eqn:potentialMethods:560}{
\begin{aligned}
&\bcF = \grad \wedge {A^\e} \\
&\grad^2 {A^\e} - \grad \lr{ \grad \cdot {A^\e} } = \eta J.
\end{aligned}
}

This can be put into correspondence with \cref{eqn:potentialMethods:120} by noting that
\begin{equation}\label{eqn:potentialMethods:460}
\begin{aligned}
\grad^2 &= (\gamma^\mu \partial_\mu) \cdot (\gamma^\nu \partial_\nu)  = \inv{v^2} \partial_{tt} - \spacegrad^2 \\
\gamma_0 {A^\e} &= \gamma_0 \gamma^\mu {A^\e}_\mu = {A^\e}_0 + \Be_k {A^\e}_k = {A^\e}_0 + \BA^\e \\
\gamma_0 \grad &= \gamma_0 \gamma^\mu \partial_\mu = \inv{v} \partial_t + \spacegrad \\
\grad \cdot {A^\e} &= \partial_\mu {A^\e}^\mu = \inv{v} \partial_t {A^\e}_0 - \spacegrad \cdot \BA^\e,
\end{aligned}
\end{equation}
so multiplying from the left with \( \gamma_0 \) gives
\begin{equation}\label{eqn:potentialMethods:480}
\lr{ \inv{v^2} \partial_{tt} - \spacegrad^2 } \lr{ {A^\e}_0 + \BA^\e } - \lr{ \inv{v} \partial_t + \spacegrad }\lr{ \inv{v} \partial_t {A^\e}_0 - \spacegrad \cdot \BA^\e } = \eta( v q_e - \bcJ ),
\end{equation}
or
\begin{subequations}
\label{eqn:potentialMethods:500}
\begin{equation}\label{eqn:potentialMethods:520}
\lr{ \inv{v^2} \partial_{tt} - \spacegrad^2 } \BA^\e - \spacegrad \lr{ \inv{v} \partial_t {A^\e}_0 - \spacegrad \cdot \BA^\e } = -\eta \bcJ
\end{equation}
\begin{equation}\label{eqn:potentialMethods:540}
\spacegrad^2 {A^\e}_0 - \inv{v} \partial_t \lr{ \spacegrad \cdot \BA^\e } = -q_e/\epsilon.
\end{equation}
\end{subequations}

So \( {A^\e}_0 = \phi \) and \( -\ifrac{\BA^\e}{v} = \bcA^\e \), or

\boxedEquation{eqn:potentialMethods:600}{
{A^\e} = \gamma_0\lr{ \phi - v \bcA^\e }.
}

\subsection{No electric sources}
Without electric sources, Maxwell's equation now splits into
\begin{subequations}
\label{eqn:potentialMethods:620}
\begin{equation}\label{eqn:potentialMethods:640}
\grad \cdot \bcF = 0
\end{equation}
\begin{equation}\label{eqn:potentialMethods:660}
\grad \wedge \bcF = -I M.
\end{equation}
\end{subequations}

Here the dual of an STA curl yields a solution
%\begin{equation}\label{eqn:potentialMethods:680}
\boxedEquation{eqn:potentialMethods:680}{
\bcF = I ( \grad \wedge {A^\m} ).
}
%\end{equation}

Substituting this gives
\begin{subequations}
\label{eqn:potentialMethods:700}
\begin{equation}\label{eqn:potentialMethods:720}
\begin{aligned}
0
&= \grad \cdot (I ( \grad \wedge {A^\m} ) ) \\
&= \gpgradeone{ \grad I ( \grad \wedge {A^\m} ) } \\
&= -I \grad \wedge ( \grad \wedge {A^\m} ).
\end{aligned}
\end{equation}
\begin{equation}\label{eqn:potentialMethods:740}
\begin{aligned}
-I M
&= \grad \wedge (I ( \grad \wedge {A^\m} ) ) \\
&= \gpgradethree{ \grad I ( \grad \wedge {A^\m} ) } \\
&= -I \grad \cdot ( \grad \wedge {A^\m} ).
\end{aligned}
\end{equation}
\end{subequations}

The \( \grad \cdot \bcF \) relation \cref{eqn:potentialMethods:720} is identically zero as desired, leaving
%\begin{equation}\label{eqn:potentialMethods:760}
\boxedEquation{eqn:potentialMethods:760}{
\grad^2 {A^\m} - \grad \lr{ \grad \cdot {A^\m} }
=
M.
}
%\end{equation}

So the general solution with both electric and magnetic sources is
%\begin{equation}\label{eqn:potentialMethods:800}
\boxedEquation{eqn:potentialMethods:800}{
\bcF = \grad \wedge {A^\e} + I (\grad \wedge {A^\m}),
}
%\end{equation}
subject to the constraints of \cref{eqn:potentialMethods:560} and \cref{eqn:potentialMethods:760}.  As before the four-potential \( {A^\m} \) can be put into correspondence with the conventional scalar and vector potentials by left multiplying with \( \gamma_0 \), which gives
\begin{equation}\label{eqn:potentialMethods:820}
\lr{ \inv{v^2} \partial_{tt} - \spacegrad^2 } \lr{ {A^\m}_0 + \BA^\m } - \lr{ \inv{v} \partial_t + \spacegrad }\lr{ \inv{v} \partial_t {A^\m}_0 - \spacegrad \cdot \BA^\m } = v q_m - \bcM,
\end{equation}
or
\begin{subequations}
\label{eqn:potentialMethods:840}
\begin{equation}\label{eqn:potentialMethods:860}
\lr{ \inv{v^2} \partial_{tt} - \spacegrad^2 } \BA^\m - \spacegrad \lr{ \inv{v} \partial_t {A^\m}_0 - \spacegrad \cdot \BA^\m } = - \bcM
\end{equation}
\begin{equation}\label{eqn:potentialMethods:880}
\spacegrad^2 {A^\m}_0 - \inv{v} \partial_t \spacegrad \cdot \BA^\m = -v q_m.
\end{equation}
\end{subequations}

Comparing with \cref{eqn:potentialMethods:260} shows that \( {A^\m}_0/v = \mu \phi_m \) and \( -\ifrac{\BA^\m}{v^2} = \mu \bcA^\m \), or

\boxedEquation{eqn:potentialMethods:900}{
{A^\m} = \gamma_0 \eta \lr{ \phi_m - v \bcA^\m }.
}
\subsection{Potential operator structure}
Observe that there is an underlying uniform structure of the differential operator that acts on the potential to produce the electromagnetic field.  Expressed as a linear operator of the
gradient and the potentials, that is
\( \bcF = L(\lrgrad, {A^\e}, {A^\m}) \)

\begin{equation}\label{eqn:potentialMethods:980}
\begin{aligned}
\bcF
&= L(\grad, {A^\e}, {A^\m}) \\
&= \grad \wedge {A^\e} + I (\grad \wedge {A^\m}) \\
&= \inv{2} \lr{ \rgrad {A^\e} - {A^\e} \lgrad } + \frac{I}{2} \lr{ \rgrad {A^\m} - {A^\m} \lgrad } \\
&= \inv{2} \lr{ \rgrad {A^\e} - {A^\e} \lgrad } + \frac{1}{2} \lr{ -\rgrad I {A^\m} - I {A^\m} \lgrad } \\
&= \inv{2} \lr{ \rgrad ({A^\e} -I {A^\m}) - ({A^\e} + I {A^\m}) \lgrad },
\end{aligned}
\end{equation}
or
%\begin{equation}\label{eqn:potentialMethods:1000}
\boxedEquation{eqn:potentialMethods:1000}{
\bcF
=
\inv{2} \lr{ \rgrad ({A^\e} -I {A^\m}) - ({A^\e} - I {A^\m})^\dagger \lgrad }
%=
%\inv{2} \lr{ \grad ({A^\e} -I {A^\m}) - (\grad ({A^\e} - I {A^\m}))^\dagger }
.
}
%\end{equation}

Observe that \cref{eqn:potentialMethods:1000} can be
put into correspondence with \cref{eqn:potentialMethods:1080} using a factoring of unity \( 1 = \gamma_0 \gamma_0 \)
\begin{equation}\label{eqn:potentialMethods:1100}
\bcF
=
\inv{2} \lr{ (-\rgrad \gamma_0) (-\gamma_0 ({A^\e} -I {A^\m})) - (({A^\e} + I {A^\m}) \gamma_0)(\gamma_0 \lgrad) },
\end{equation}
where
\begin{subequations}
\label{eqn:potentialMethods:1120}
\begin{equation}\label{eqn:potentialMethods:1140}
\begin{aligned}
-\grad \gamma_0
&= -(\gamma^0 \partial_0 + \gamma^k \partial_k) \gamma_0 \\
&= -\partial_0 - \gamma^k \gamma_0 \partial_k \\
&= \spacegrad -\inv{v} \partial_t,
\end{aligned}
\end{equation}
\begin{equation}\label{eqn:potentialMethods:1160}
\begin{aligned}
\gamma_0 \grad
&= \gamma_0 (\gamma^0 \partial_0 + \gamma^k \partial_k) \\
&= \partial_0 - \gamma^k \gamma_0 \partial_k \\
&= \spacegrad + \inv{v} \partial_t,
\end{aligned}
\end{equation}
\end{subequations}
and
\begin{subequations}
\label{eqn:potentialMethods:1180}
\begin{equation}\label{eqn:potentialMethods:1200}
\begin{aligned}
-\gamma_0 ( {A^\e} - I {A^\m} )
&= -\gamma_0 \gamma_0 \lr{ \phi -v \bcA^\e + \eta I \lr{ \phi_m - v \bcA^\m } } \\
&= -\lr{ \phi -v \bcA^\e + \eta I \phi_m - \eta v I \bcA^\m } \\
&= - \phi + v \bcA^\e + \eta v I \bcA^\m - \eta I \phi_m
\end{aligned}
\end{equation}
\begin{equation}\label{eqn:potentialMethods:1220}
\begin{aligned}
( {A^\e} + I {A^\m} )\gamma_0
&= \lr{ \gamma_0 \lr{ \phi -v \bcA^\e } + I \gamma_0 \eta \lr{ \phi_m - v \bcA^\m } } \gamma_0 \\
&= \phi + v \bcA^\e + I \eta \phi_m + I \eta v \bcA^\m \\
&= \phi + v \bcA^\e + \eta v I \bcA^\m + \eta I \phi_m,
\end{aligned}
\end{equation}
\end{subequations}

This recovers \cref{eqn:potentialMethods:1080} as desired.
