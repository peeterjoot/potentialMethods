%
% Copyright © 2017 Peeter Joot.  All Rights Reserved.
% Licenced as described in the file LICENSE under the root directory of this GIT repository.
%

\subsection{No magnetic sources}

When magnetic sources are omitted, it follows from \cref{eqn:chapter3Notes:80} that there is some \( \bcA^\e \) for which

%\begin{equation}\label{eqn:potentialMethods:20}
\boxedEquation{eqn:potentialMethods:20}{
\bcB = \spacegrad \cross \bcA^\e,
}
%\end{equation}

Substitution into Faraday's law \cref{eqn:chapter3Notes:20} gives

\begin{equation}\label{eqn:potentialMethods:40}
\spacegrad \cross \bcE = - \PD{t}{}\lr{ \spacegrad \cross \bcA^\e },
\end{equation}

or
\begin{equation}\label{eqn:potentialMethods:60}
\spacegrad \cross \lr{ \bcE + \PD{t}{ \bcA^\e } } = 0.
\end{equation}

A gradient representation of this curled quantity, say \( -\spacegrad \phi \), will provide the required zero

%\begin{equation}\label{eqn:potentialMethods:80}
\boxedEquation{eqn:potentialMethods:80}{
\bcE = -\spacegrad \phi -\PD{t}{ \bcA^\e }.
}
%\end{equation}

The final two Maxwell equations yield

\begin{equation}\label{eqn:potentialMethods:100}
\begin{aligned}
-\spacegrad^2 \bcA^\e + \spacegrad \lr{ \spacegrad \cdot \bcA^\e } &= \mu \lr{ \bcJ + \epsilon \PD{t}{} \lr{ -\spacegrad \phi -\PD{t}{ \bcA^\e } } } \\
\spacegrad \cdot \lr{ -\spacegrad \phi -\PD{t}{ \bcA^\e } } &= q_e/\epsilon,
\end{aligned}
\end{equation}

or
%\begin{equation}\label{eqn:potentialMethods:120}
\boxedEquation{eqn:potentialMethods:120}{
\begin{aligned}
\spacegrad^2 \bcA^\e - \inv{v^2} \PDSq{t}{ \bcA^\e }
- \spacegrad \lr{
\inv{v^2} \PD{t}{\phi}
+\spacegrad \cdot \bcA^\e
}
&= -\mu \bcJ \\
\spacegrad^2 \phi + \PD{t}{} \lr{ \spacegrad \cdot \bcA^\e } &= -q_e/\epsilon.
\end{aligned}
}
%\end{equation}

Note that the Lorentz condition \( \PDi{t}{(\phi/v^2)} + \spacegrad \cdot \bcA^\e = 0 \) can be imposed to decouple these, leaving non-homogeneous wave equations for the vector and scalar potentials respectively.

\subsection{No electric sources}

Without electric sources, a curl representation of the electric field can be assumed, satisfying Gauss's law

\boxedEquation{eqn:potentialMethods:140}{
\bcD = - \spacegrad \cross \bcA^\m.
}

Substitution into the Maxwell-Faraday law gives
\begin{equation}\label{eqn:potentialMethods:160}
\spacegrad \cross \lr{ \bcH + \PD{t}{\bcA^\m} } = 0.
\end{equation}

This is satisfied with any gradient, say, \( -\spacegrad \phi_m \), providing a potential representation for the magnetic field

%\begin{equation}\label{eqn:potentialMethods:180}
\boxedEquation{eqn:potentialMethods:180}{
\bcH = -\spacegrad \phi_m - \PD{t}{\bcA^\m}.
}
%\end{equation}

The remaining Maxwell equations provide the required constraints on the potentials

\begin{subequations}
\label{eqn:potentialMethods:200}
\begin{equation}\label{eqn:potentialMethods:220}
-\spacegrad^2 \bcA^\m + \spacegrad \lr{ \spacegrad \cdot \bcA^\m } = -\epsilon
\lr{
   -\bcM - \mu \PD{t}{}
   \lr{
      -\spacegrad \phi_m - \PD{t}{\bcA^\m}
   }
}
\end{equation}
\begin{equation}\label{eqn:potentialMethods:240}
\spacegrad \cdot
\lr{
-\spacegrad \phi_m - \PD{t}{\bcA^\m}
}
= \inv{\mu} q_m,
\end{equation}
\end{subequations}

or
%\begin{equation}\label{eqn:potentialMethods:260}
\boxedEquation{eqn:potentialMethods:260}{
\begin{aligned}
\spacegrad^2 \bcA^\m - \inv{v^2} \PDSq{t}{\bcA^\m} - \spacegrad \lr{ \inv{v^2} \PD{t}{\phi_m} + \spacegrad \cdot \bcA^\m } &= -\epsilon \bcM \\
\spacegrad^2 \phi_m + \PD{t}{}\lr{ \spacegrad \cdot \bcA^\m } &= -\inv{\mu} q_m.
\end{aligned}
}
%\end{equation}

The general solution to Maxwell's equations is therefore
\begin{equation}\label{eqn:potentialMethods:280}
\begin{aligned}
\bcE &=
-\spacegrad \phi -\PD{t}{ \bcA^\e }
- \inv{\epsilon} \spacegrad \cross \bcA^\m \\
\bcH &=
\inv{\mu} \spacegrad \cross \bcA^\e
-\spacegrad \phi_m - \PD{t}{\bcA^\m},
\end{aligned}
\end{equation}

subject to the constraints \cref{eqn:potentialMethods:120} and \cref{eqn:potentialMethods:260}.

\subsection{Potential operator structure}

Knowing that there is a simple underlying structure to the potential representation of the electromagnetic field in the STA formalism inspires the question of whether that structure can be found directly using the scalar and vector potentials determined above.

Specifically, what is the multivector representation \cref{eqn:potentialMethods:1020} of the electromagnetic field in terms of all the individual potential variables, and can an underlying structure for that field representation be found?  The composite field is

\begin{equation}\label{eqn:potentialMethods:280b}
\bcF
=
-\spacegrad \phi -\PD{t}{ \bcA^\e }
- \inv{\epsilon} \spacegrad \cross \bcA^\m \\
+ I \eta
\lr{
\inv{\mu} \spacegrad \cross \bcA^\e
-\spacegrad \phi_m - \PD{t}{\bcA^\m}
}.
\end{equation}

Can this be factored into into multivector operator and multivector potentials?  Expanding the cross products provides some direction

\begin{equation}\label{eqn:potentialMethods:1040}
\begin{aligned}
\bcF
&=
- \PD{t}{ \bcA^\e }
- \eta \PD{t}{I \bcA^\m}
- \spacegrad \lr{ \phi - \eta I \phi_m } \\
&\quad + \frac{\eta}{2 \mu} \lr{ \rspacegrad \bcA^\e - \bcA^\e \lspacegrad }
+ \frac{1}{2 \epsilon} \lr{ \rspacegrad I \bcA^\m - I \bcA^\m \lspacegrad }.
\end{aligned}
\end{equation}

Observe that the
gradient and the time partials can be grouped together

\begin{equation}\label{eqn:potentialMethods:1060}
\begin{aligned}
\bcF
&=
- \PD{t}{ } \lr{\bcA^\e + \eta I \bcA^\m}
- \spacegrad \lr{ \phi + \eta I \phi_m }
+ \frac{v}{2} \lr{ \rspacegrad (\bcA^\e + I \eta \bcA^\m) - (\bcA^\e + I \eta \bcA^\m) \lspacegrad } \\
%=
%- \inv{v} \PD{t}{ (v \bcA^\e) } + \frac{1}{2} \lr{ \rspacegrad (v \bcA^\e) - (v \bcA^\e) \lspacegrad }
%- \inv{v} \PD{t}{} (\eta v I \bcA^\m)
%+ \frac{1}{2 } \lr{ \rspacegrad (\eta v I \bcA^\m) - (\eta v I \bcA^\m) \lspacegrad }
%- \spacegrad \lr{ \phi + \eta I \phi_m }
&=
\inv{2} \lr{
   \lr{ \rspacegrad - \inv{v} \rpartialt } \lr{ v \bcA^\e + \eta v I \bcA^\m }
   -
   \lr{ v \bcA^\e + \eta v I \bcA^\m} \lr{ \lspacegrad + \inv{v} \lpartialt }
} \\
&+\quad \inv{2} \lr{
   \lr{ \rspacegrad - \inv{v} \rpartialt } \lr{ -\phi - \eta I \phi_m }
   - \lr{ \phi + \eta I \phi_m } \lr{ \lspacegrad + \inv{v} \lpartialt }
}
% \\
%&=
%- \inv{2} \Biglr{
%   \lr{ \rspacegrad - \inv{v} \rpartialt } \lr{ - v \bcA^\e - \eta v I \bcA^\m +  \phi + \eta I \phi_m } \\
%   &\qquad -
%   \lr{ - v \bcA^\e - \eta v I \bcA^\m - \phi - \eta I \phi_m } \lr{ \lspacegrad + \inv{v} \lpartialt }
%}
,
\end{aligned}
\end{equation}

or

%\begin{equation}\label{eqn:potentialMethods:1080}
\boxedEquation{eqn:potentialMethods:1080}{
\begin{aligned}
\bcF
&=
\inv{2} \Biglr{
   \lr{ \rspacegrad - \inv{v} \rpartialt }
   \lr{
      - \phi
      + v \bcA^\e
      + \eta I v \bcA^\m
      - \eta I \phi_m
   } \\
   &\qquad -
   \lr{
      \phi
      + v \bcA^\e
      + \eta I v \bcA^\m
      + \eta I \phi_m
   }
   \lr{ \lspacegrad + \inv{v} \lpartialt }
}
.
\end{aligned}
}
%\end{equation}

There's a conjugate structure to the potential on each side of the curl operation where we see a sign change for the scalar and pseudoscalar elements only.  The reason for this becomes more clear in the STA formalism.

