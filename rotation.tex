%
% Copyright � 2017 Peeter Joot.  All Rights Reserved.
% Licenced as described in the file LICENSE under the root directory of this GIT repository.
%
%{
\input{../latex/blogpost.tex}
\renewcommand{\basename}{rotation.tex}
%\renewcommand{\dirname}{notes/phy1520/}
\renewcommand{\dirname}{notes/ece1228-electromagnetic-theory/}
%\newcommand{\dateintitle}{}
%\newcommand{\keywords}{}

\input{../latex/peeter_prologue_print2.tex}

\usepackage{peeters_layout_exercise}
\usepackage{peeters_braket}
\usepackage{peeters_figures}
\usepackage{siunitx}
%\usepackage{mhchem} % \ce{}
%\usepackage{macros_bm} % \bcM
%\usepackage{macros_qed} % \qedmarker
%\usepackage{txfonts} % \ointclockwise

\beginArtNoToc

\generatetitle{Change of basis}
%\chapter{Change of basis}
%\label{chap:rotation.tex}

Given two bases
\( \beta_1 = \setlr{ \Be_1, \Be_2, \Be_3 } \), and
\( \beta_2 = \setlr{ \Bf_1, \Bf_2, \Bf_3 } \),
three multivector rotation operators are required to make a change of basis transformation, say

\begin{equation}\label{eqn:rotation:20}
\Bf_k = \Be_k e^{B_k} = e^{-B_k} \Be_k.
\end{equation}

Here \( B_k \) is a bivector for which the orientation is the plane of rotation between \( \Be_k \) and \( \Bf_k \)

\begin{equation}\label{eqn:rotation:40}
\hat{B}_k = \frac{B_k}{\Abs{B_k}} = \frac{\Be_k \wedge \Bf_k}{\Abs{\Be_k \wedge \Bf_k}},
\end{equation}

and has the magnitude \( \Abs{B_k} = \theta_k \), the rotation angle between \( \Be_k \) and \( \Bf_k \).

If the coordinate representation in the two bases is given by

\begin{equation}\label{eqn:rotation:60}
\Bx = \sum_k x_k \Be_k = \sum_k y_k \Bf_k,
\end{equation}

then the coordinates can be related by the same transformation.  That is

\begin{equation}\label{eqn:rotation:80}
\begin{aligned}
y_i
&= \Bx \cdot \Bf_i \\
&= \gpgradezero{
\lr{ \sum_j x_j \Be_j } \Bf_i
} \\
&=
\sum_j x_j
\gpgradezero
{
\Be_j \Be_i e^{B_j}
} \\
&=
\sum_j x_j \lr{ (\Be_j \cdot \Be_i) \cosh(B_j) + (\Be_j \wedge \Be_i) \cdot \sinh(B_j) } \\
&=
\sum_j x_j \lr{ (\Be_j \cdot \Be_i) \cos(\theta_j) + ((\Be_j \wedge \Be_i) \cdot \hat{B}_j) \sin(\theta_j) },
\end{aligned}
\end{equation}

If desired the explicit expansion of the
bivector dot product factor of the sine is

\begin{equation}\label{eqn:rotation:100}
\begin{aligned}
(\Be_j \wedge \Be_i) \cdot \hat{B}_j
&=
(\Be_j \wedge \Be_i) \cdot \frac{\Be_j \wedge \Bf_j}{\sqrt{-(\Be_j \wedge \Bf_j) \cdot (\Be_j \wedge \Bf_j)}} \\
=
\frac{
   (\Be_j \cdot \Bf_j)(\Be_i \cdot \Be_j)
   -
   (\Be_j \cdot \Be_j)(\Be_i \cdot \Bf_j)
}{
\sqrt{
(\Be_j \cdot \Be_j)(\Bf_j \cdot \Bf_j)
-(\Bf_j \cdot \Be_j)(\Be_j \cdot \Bf_j)
}
}.
\end{aligned}
\end{equation}

The change of basis can be expressed as a rotation matrix with coordinates \( R_{ij} \) given by
\begin{equation}\label{eqn:rotation:120}
R_{ij}
=
(\Be_j \cdot \Be_i) \cos(\theta_j) + ((\Be_j \wedge \Be_i) \cdot \hat{B}_j) \sin(\theta_j),
\end{equation}
where the coordinate transformation is
\begin{equation}\label{eqn:rotation:140}
y_i = R_{ij} x_j.
\end{equation}
%}
%\EndArticle
\EndNoBibArticle
